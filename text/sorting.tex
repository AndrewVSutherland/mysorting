\documentclass{article}
\usepackage{amsmath, amsfonts, amsthm, amssymb}
%\usepackage{sagetex}
\usepackage{comment}
\usepackage{url}

\newtheorem{thm}{Theorem}[section]
\newtheorem{cor}[thm]{Corollary}
\newtheorem{lem}[thm]{Lemma}
\newtheorem{prop}[thm]{Proposition}
\newtheorem{defn}{Definition}%[section]
\newtheorem{rem}[thm]{Remark}

\def\Sage{{\tt Sage}}
\def\Z{{\mathbb Z}}
\def\Q{{\mathbb Q}}
\def\R{{\mathbb R}}
\def\C{{\mathbb C}}
\def\T{{\mathbb T}}
\def\H{{\mathbb H}}
%\def\RC{{K_\infty}}
\def\RC{{\tilde{K}}}
\def\P{{\mathbb P}}
\def\F{{\mathbb F}}
\def\Fp{{\mathbb F}_p}
\def\Fq{{\mathbb F}_q}
\def\M{{\mathcal M}}
\def\CC{{\mathcal C}}
\def\HH{{\mathcal H}}
\def\SS{{\mathcal S}}
\def\G{{\mathcal G}}
\def\K{{\mathcal K}}
\def\I{{\mathcal I}}
\def\J{{\mathcal J}}
\def\PP{{\mathcal P}}
\def\OO{{\mathcal O}}
\def\ZG{{\mathcal Z}}
\def\a{{\mathfrak a}}
\def\b{{\mathbf b}}

\def\v{{\mathbf v}}
\def\c{{\mathfrak c}}
\def\d{{\mathfrak d}}
\def\m{{\mathfrak m}}
\def\n{{\mathfrak n}}
\def\p{{\mathfrak p}}
\def\q{{\mathfrak q}}
\def\r{{\mathfrak r}}
\def\DF{{\nabla\underline{F}}}
\def\Qbar{\overline{\Q}}
\def\Qpbar{\overline{\Q_p}}
\def\Abar{\overline{A}}
\def\RR{{R\oplus R}}
\def\KK{{K\oplus K}}
\def\uF{{\underline{F}}}
\def\oC{\overline{\mathcal C}}
\def\oP{\overline{P}}
\def\oM{\overline{M}}
\DeclareMathOperator{\lcm}{lcm}
\DeclareMathOperator{\disc}{disc}
\DeclareMathOperator{\rk}{rank}
\DeclareMathOperator{\ord}{ord}
\DeclareMathOperator{\diag}{diag}
\DeclareMathOperator{\adj}{adj}
\DeclareMathOperator{\Mat}{Mat}
\DeclareMathOperator{\Hom}{Hom}
\DeclareMathOperator{\Res}{Res}
\DeclareMathOperator{\Gal}{Gal}
\DeclareMathOperator{\End}{End}
\DeclareMathOperator{\Cl}{Cl}
\def\<#1>{\left<#1\right>}
\newcommand{\mat}[4]{\left(\begin{matrix} %
                   #1 & #2 \\ #3 & #4 %
                  \end{matrix}\right)}
\newcommand{\smat}[4]{\left(\begin{smallmatrix} %
                   #1 & #2 \\ #3 & #4 %
                  \end{smallmatrix}\right)}
\DeclareMathOperator{\im}{im}


\title{Notes on sorting number field elements, prime ideals and
  isogenous elliptic curves over number fields} \author{John Cremona}

\begin{document}
\maketitle

These notes concern algorithms for sorting different objects:
algebraic numbers, prime ideals of a fixed number field, and elliptic
curves in an isogeny class over a number field.

\section{Sorting number field elements}
\subsection{Defining the number field}
We assume that every number field $K$ is given in the form
$K=\Q(\alpha)$ where the minimal polynomial $f(X)$ of $\alpha$ is
canonically determined by the field $K$.  In practice, $f(X)$ is the
polynomial returned by the {\tt Pari} function {\tt polredabs} which
is guaranteed to always return the same polynomial, given any
polynomial defining a number field isomorphic to $K$.  This canonical
polynomial $f(X)$ is thus fixed one and for all; it is monic and
integral.

We do not assume that $\Z[\alpha]$ is the full ring of integers
$\OO_K$, since in any case there exist number fields whose rings of
integers are not monogenic over $\Z$.

We emphasise that both the defining polynomial $f(X)$ and the generator
$\alpha$ of~$K$ are fixed.  In other words we are fixing the structure
of $K$ not just as a field but as a $\Q[X]$-algebra, where the
structure map $\Q[X]\to K$ has kernel $(f(X))$ and $\alpha$ is the
image of $X$.

\subsection{Sorting elements of $K$}
Given $K=\Q(\alpha)$ as above, with degree $d=[K:\Q]=\deg f$, every
element $\beta\in K$ has a unique expression as a polynomial in
$\alpha$ of degree at most $d-1$, say
$\beta=\sum_{i=0}^{d-1}b_i\alpha^i$ and we use the coefficient vector
$(b_0,b_1,\dots,b_{d-1})\in\Q^d$ as a sorting key, where $\Q^d$ is
sorted lexicographically.  To remove any ambiguity, if also
$\gamma=\sum_{i=0}^{d-1}c_i\alpha^i$ then $\beta<\gamma$ if and only
if for the \textit{smallest} index~$i$ for which $b_i\not=c_i$ we have
$b_i<c_i$.

As a corollary, the induced ordering on $\Q$ viewed as a subset of $K$
is the natural one.

\subsection{An incomplete alternative sorting method}

The sorting of algebraic numbers defined above relies on all numbers
to be sorted lying in one fixed number field, with a fixed generator,
and the ordering will depend on this common ambient field.  For
example, two elements of $K=\Q(\sqrt{2})$ may have one order in $K$
but a different order in a larger field such as
$\Q(\sqrt{2},\sqrt{3})$.

It is tempting to try to define an ordering on all of $\Qbar$ which
relies on no such choices.  Every $\beta\in\Qbar$ is a root of a
unique polynomial $g(X)\in\Z[X]$ which is primitive and irreducible
and with positive leading coefficient.  Using the coefficient sequence
of $g$ as a key allows comparison of any two non-conjugate algebraic
numbers.  Numbers of lower degree come before numbers of higher
degree; for those of the same degree, if the leading coefficient was
given priority, this ordering would even place all algebraic integers
before all non-integers.

However, in order to be completely deterministic we would have to
define an ordering of each set of Galois conjugates, which share the
same polynomial $g(X)$.  This could be done, for example, by fixing an
embedding of $\Qbar$ into $\C$ and using lexicgraphical ordering of
real and imaginary parts on $\C$.  But then to apply the ordering to
elements of a number field $K=\Q(\alpha)$ would require fixing for
each such field an embedding into $\Qbar$, with these embeddings being
compatible with respect to inclusions.  This may well be possible
though not easy to implement.

On balance we decided to use the first sorting method based on the
power basis coordinates in a common fixed ambient field (with fixed
defining polynomial and generator).

\section{Sorting prime ideals}
Fix a number field $K=\Q(\alpha)$ with defining polynomial
$f(X)\in\Z[X]$ and generator $\alpha$ as above.  We do not assume that
$K/\Q$ is Galois.

We wish to define a linear order on the set of all prime ideals~$\p$
of $K$.  Denote by $p$ the underlying rational prime, $e=e(\p/p)$ the
ramification index and $f=f(\p/p)$ the residue field degree
$[\OO_K/\p:\F_p]$.  The norm is $N(\p)=p^f$.

In \Sage, if $\p=${\tt P} is the prime ideal then its norm is
$N(\p)=${\tt P.norm()}, the underlying prime is $p=${\tt
  P.smallest\_integer()}, the residue degree is $f=${\tt
  P.residue\_class\_degree()} (so that $N(\p)=p^f$), and the
ramification index is $e=${\tt P.ramification\_index()}.

We first sort by the norm, then by the ramification index. Now we have
to sort primes with the same value of $p^{ef}$.

\subsection{The case of unramified primes}

Our general method is simplest when $p$ is unramified, not just in $K$
but in the order $\Z[\alpha]$, i.e. when $p\nmid\disc(f)$.  In this
case $f(X)$ factors modulo~$p$ into distinct irreducibles and we may
write
\[
    f \equiv \prod_{i=1}^{g}h_i(X) \pmod{p},
\]
with the factors $h_i(X)\in\Z[X]$ monic and such that their reductions
modulo~$p$ are distinct and irreducible.  We may assume that the
coeffients of each $h_i(X)$ are reduced modulo~$p$ to lie in
$\{0,1,\dots,p-1\}$, and we order these factors first by degree and
then by lexicographical ordering of the coefficients.

To each factor $h_i(X)$ we associate the ideal $\p_i=(p,h_i(\alpha))$
which has $f(\pi_i/p)=\deg h_i$, and the ordering of the $h_i$ thus
induces the desired ordering of the primes $\p_i$ with the same
degree.

To compute the associated sorting key attahed to a prime~$\p$
above~$p$: factor $f(X)$ modulo~$p$ as above, let $h(X)$ be the unique
$h_i(X)$ such that $\ord_{\p}(h_i(\alpha))>0$; then the key is the
coefficient vector of $h_i$ in $\F_p^f$.

\subsection{The general case, including ramified and unramified primes}

There is a bijection between the distinct primes $\p$ above $p$ and
the irreducible factors $h(X)$ of $f(X)$ over $\Q_p$, such that $\deg h
= e(\p/p)f(\p/p)$.  Write
\[
  f(X) = h_1(X)h_2(X)\dots h_g(X)
\]
with the $h_i(X)$ monic and irreducible in $\Z_p[X]$, sorted as
follows: first by degree, then by lexicographical order of the
coefficient vectors, where $\Z_p$ is ordered lexicographically by
$p$-adic digits.  In what follows it will be sufficient to know the
$h_i$ to finite $p$-adic precision.

Let $k\ge1$ be an integer such that the $h_i(X)$ are distinct
modulo~$p^k$.  We may take $k=1$ if and only if $p$ is unramified (in
the sense that it does not divide $\disc f$).  Denote by $h_i^k(X)$
the unique monic polynomial in $\Z[X]$ with coefficients $c$
satisfying $0\le c\le p^k-1$ which is congruent to $h_i(X)\pmod{p^k}$.
Then the ordering of the $h_i(X)$ of the same degree defined in the
previous paragraph as the same as the ordering of the $h_i^k(X)$ by
lexicographical ordering of their integer coefficient vectors.

For each prime $\p\mid p$ there is a unique index~$i$ such that
$\ord_{\p}(h_i^k(\alpha))$ is strictly greater than all other
$\ord_{\p}(h_j^k(\alpha))$.  For example, in the unramified case when
$k=1$, $\ord_{\p}(h_i^1(\alpha))>0$ for one value of~$i$, while
$\ord_{\p}(h_j^1(\alpha))=0$ for $j\not=i$.  We prove this as follows:

For each $i$ let $\alpha_i\in\Qpbar$ be a root of $h_i(X)$.  There are
$g$ different embeddings of $K$ into $\Qpbar$, up to conjugacy,
defined by mapping $\alpha$ to $\alpha_i$ for $1\le i\le g$.  Since
$h_i(\alpha_i)=0$ while $h_i(\alpha_j)\not=0$ for $j\not=i$, we have
that $\lim_{k\to\infty}h_i^k(\alpha_i)=0$ while
$\lim_{k\to\infty}h_i^k(\alpha_j)\not=0$ for $j\not=0$.

Each field $\Q_p(\alpha_i)$ is the completions $K_{\p_i}$ of $K$ at
some prime~$\p_i$ above~$p$, and the absolute value in each
$\Q_p(\alpha_i)$ is that induced by the $\p_i$-adic valuation on $K$
and on $\Q_p(\alpha_i)$.  Hence the limits in the previous paragraph
are equivalent to saying that
\[
    \lim_{k\to\infty}\ord_{\p_i}(h_i^k(\alpha)) = \infty,
\]
while $\ord_{\p_j}(h_i^k(\alpha))$ remains bounded as $k\to\infty$ for
$j\not=i$.  Thus for large enough $k$, $\ord_{\p_i}(h_i^k(\alpha))$ is
strictly greater that $\ord_{\p_i}(h_j^k(\alpha))$ for all $j\not=i$.
We can use this fact to uniquely assigned one of the $h_i^k$ to each
$\p_i$ and use the ordering of the $h_i$ defined above to order the
$\p_i$.

In order to implement this in practice we need to know how large value
of $k$ is needed.  Is it sufficient for $k$ to be such that the sets
$\{\ord_{\p_i}(h_j^k(\alpha)) \mid 1\le j\le g\}$ have a unique
maximal element, for all~$i$?  If so we can start with some $k$ and
increase it until this property holds.  This will be easier than
deciding in advance which $k$ work, in terms of the $p$-adic valuation
of $\disc f$ perhaps.

\subsection{Example}
We take an example from
\url{http://wstein.org/129/ant/html/node29.html} where $K$ is a
non-Galois cubic with $2$ as an essential discriminant divisor.  Let
$f(X) = X^3+X^2-2X+8$, with discriminant $\disc f = -2^2\cdot503$.

Let $p=2$ and $k=2$; the $2$-adic factors of $f$ are, reduced
modulo~$4$: $h_1^2=X$, $h_2^2=X+2$, $h_3^2=X+3$.  So there are $3$
primes above $2$ (which is in fact unramified in $K$).  Calling these
$\p_a$, $\p_b$, $\p_c$ in random order we find that
\begin{itemize}
\item the $\p_a$-valuations of the $h_i^2$ are $(1,2,0)$
\item the $\p_b$-valuations of the $h_i^2$ are $(0,0,2)$
\item the $\p_c$-valuations of the $h_i^2$ are $(2,1,0)$
\end{itemize}
and hence $\p_1=\p_c$, $\p_2=\p_a$ and $\p_3=\p_b$.

Let $p=503$.  There are two primes above~$p$ both of norm~$p$, only
one of which is ramified (with $e=2$).  In our ordering we put the
unramified one first and the ramified one second, so we do not have to
look at the $p$-adic factorization of $f(X)$ in this case.  In fact,
with $k=2$ we find factors $X+170313$ and $X^2+82697X+130849$.

\section{Sorting all integral ideals}

Work in progress.

\section{Sorting isogenous elliptic curves}

\subsection{Elliptic curve labels}

Labels for elliptic curves over number fields (including $\Q$ have
three components: a label for the conductor, a label for each isogeny
class with that conductor, and a label for each curve in its isogeny
class.

Since conductors are integral ideals we can label them according to
the previous section.

For a fixed conductor each isogeny class is uniquely determined by the
traces of Frobenius $a_{\p}(E)$.  Here we include primes of bad
reduction in the usual way: $a_{\p}(E)=+1$ for split multiplicative
reduction, $-1$ for non-split and $0$ for additive reduction.  Now we
can sort the isogeny classes of any fixed conductor by
lexicographically ordering the sequence of $a_{\p}(E)$ with the
primes~$\p$ ordered as in the previous section.  Having done that,
each isogeny class has an index in the order; we base the indices
at~$0$ and convert to base 26 using ``digits'' $a=0$, $b=1$, $\dots$,
$z=25$, obtaining a string from the list $a, b, \dots, z, ba, bb,
\dots$.

Remarks: 1. In the LMFDB at present (June 2016) the isogeny class
labels follow this scheme for imaginary quadratic fields since there
the primes are sorted as in the previous section.  However the class
labels for elliptic curves over totally real fields are taken from the
labels of the associated Hilbert modular forms , whose ordering is not
necessary well-defined as described above.  This is a historical
accident.

2. Over $\Q$ the LMFDB labels for elliptic curves follow this but the
Cremona labels do not for conductors up to $230000$.  Again this is a
historical phenomenon.

Finally, the curves in each isogeny class are sorted in some way and
hence given a numerical index, starting at 1, which forms the third
part of the label.  In the rest of this section we describe methods
used to sort the curves in each isogeny class.

\subsection{Over $\Q$}

Every elliptic curve over $\Q$ has a unique Weierstrass model which is
integral, globally minimal and with coefficients $a_1,a_3\in\{0,1\}$
and $a_3\in\{-1,0,1\}$.  We sort the curves using this coefficient
vector $(a_1,a_2,a_3,a_4,a_6)$.  In fact, within on isogeny class the
triple $(a_1,a_2,a_3)$ is constant\footnote{Siksek, Samir \textit{On
    standardized models of isogenous elliptic curves}. Math. Comp. 74
  (2005), no. 250, 949–951} so it suffices to sort on $(a_4,a_6)$.

This is the sorting used for the LMFDB labels for elliptic curves over
$\Q$.  It differs from the Cremona labelling, in which the ``strong
Weil'' or $\Gamma_0(N)$-optimal curve is number~$1$ and the others are
labelled in the order of being computed from the first.  At some point
the latter became completely deterministic and has been documented
elsewhere, but the details are not relevant here.

There are many reasons why a system of ordering isogenous curves using
Weierstrass models is hopeless over almost all number fields (the only
exceptions, perhaps, being the $7$ imaginary quadratic fields having
only $\pm1$ as units, and class number~$1$).  There may be no global
minimal model, in which case some ``semi-global minimal'' model is
stored in the LMFDB; even when there is a global minimal model,
scaling by units results in infinitely many such models, even up to
translation (except for imaginary quadratic fields where the number is
finite).

Instead we have devised a system of sorting isogenous curves which
does not depend on the model, based on the simple observation (see the
next subsection) that in almost all cases isogenous curves have
distinct $j$-invariants.  It would be possible to use this new system
over $\Q$, but as there are already two labelling systems we do not
propose this!

\subsection{Over arbitrary number fields}

The basic fact we need is that in almost all cases, the elliptic
curves in an isogeny class have distinct $j$-invariants.

\begin{thm}
Let $K$ be a number field and $E_1$, $E_2$ elliptic curves defined
over~$K$ which are isogenous over~$K$ but not isomorphic over~$K$.
Then either
\begin{enumerate}
\item $j(E_1)\not=j(E_2)$; or
\item $E_1$ and $E_2$ have CM by some imaginary quadratic order $\OO$
  of discriminant~$D<0$ such that $\sqrt{D}\notin K$ (so that the extra
  endomorphisms are not defined over~$K$), $E_1$ and $E_2$ become
  isomorphic over $L=K(\sqrt{D})$, and there are cyclic isogenies
  between $E_1$ and $E_2$ of degree $|D|$ (if $D$ is odd), $|D|/4$
  (if $D$ is even and $D\not=-4$) or~$2$ (if $D=-4$).
\end{enumerate}
Conversely if $E$ is an elliptic curve defined over $K$ with CM by an
order of discriminant~$D$ not contained in $K$, then the quadratic
twist of $E$ by $\sqrt{D}$ is isogenous but not isomorphic to~$E$
over~$K$, and hence the isogeny class of $E$ consists of pairs of
curves, the curves in each pair being quadratic twists of each other
over $K(\sqrt{D})$.
\end{thm}

\begin{proof}
Let $\phi:E_1\to E_2$ be an isogeny defined over~$K$; without loss of
generality we may take $\phi$ to be cyclic (since otherwise it factors
through the multiplication-by-$m$ map on $E_1$ for some $m>1$).

Assume that $j(E_1)=j(E_2)$.  Then there is an isomorphism
$\alpha:E_2\to E_1$ defined over an extension $L$ of $K$ of degree at
most~$6$, and no such isomorphism defined over $K$ itself, by
assumption.

The composite $\psi=\alpha\circ\phi$ is a cyclic endomorphism of
$E_1$, showing that $E_1$ (and hence also $E_2$) has CM by some
imaginary quadratic order~$\OO$, of discriminant $D$, say.  Now $\psi$
is defined over $L=K(\sqrt{D})$, and $L$ contains the field of
definition of~$\alpha$, showing that $[L:K]=2$ and that $\alpha$ is
defined over~$L$.  Hence $E_1$ and $E_2$ become isomorphic over the
quadratic extension~$L$.

Let $\sigma$ denote the nontrivial element of $\Gal(L/K)$.

Assume first that $j\not=0, 1728$, or equivalently that $D\not=-3,-4$.
Then $\alpha^{\sigma}=-\alpha$, hence $\psi^{\sigma}=-\psi$.
Moreover, the fact that $\psi$ has cyclic kernel implies that $\psi$
is not divisible in $\End(E_1)$ by any integer $n>1$.  If $D$ is odd
so that $\OO=\Z[(1+\sqrt{D})/2]$, this implies that
$\psi=\pm\sqrt{D}$, so $\deg\psi=|D|$.  If $D$ is even then
$\OO=\Z[(1+\sqrt{D})/2]$ and $\psi=\pm\sqrt{D}/2$ with degree $|D|/4$.

Suppose that $j=0$ and $D=-3$.  Without loss we may assume that $E_1$
has equation $Y^2=X^3+a$ with $a\in K^*$; since $E_2$ becomes
isomorphic to $E_1$ over the quadratic extension $K(\sqrt{-3})$, it is
isomorphic to $Y^2=X^3+a'$ where $a'/a$ represents a nontrivial
element of the kernel of the map $K^*/(K^*)^6 \to L^*/(L^*)^6$, but
the only such element (up to 6th powers) is $-27$ (easy exercise).
Hence without loss of generality $E_2$ has equation $Y^2=X^3-27a$, and
standard formulas show that $E_1$ and $E_2$ are $3$-isogenous.  To see
the last part directly, one isomorphism $E_1\to E_2$ is given by
$(x,y)\mapsto(-3x,-3\sqrt{-3}y)$ so that as in the generic case
$\alpha^{\sigma}=-\alpha$ and $\psi$ is pure imaginary, hence
$\psi=\pm\sqrt{-3}$ with degree~$3$.

Finally suppose that $j=1728$ and $D=-4$, so $L=K(\sqrt{-1})$.  The
kernel of $K^*/(K^*)^4 \to L^*/(L^*)^4$ has order~$2$ with non-trivial
element represented by~$-4$, and hence without loss of generality we
may suppose that $E_1$ has equation $Y^2=X^3+aX$ and $E_2:
Y^2=X^3-4aX$ for some $a\in K^*$.  Again it is well-known that these
are $2$-isogenous.  Alternatively, an explicit computation (similar to
that for $j=0$) shows that $\psi=\pm1\pm i$ with degree~$2$; note that
in this case $\alpha^{\sigma}\not=\pm\alpha$.

For the last part let $E$ have CM by the order $\OO$ of
discriminant~$D<0$ with $\sqrt{D}\notin K$.  Set $L=K(\sqrt{D})$ and
let $E'$ be the $\sqrt{D}$-twist of $E$.  Assume first that
$D\not=-4$; then $\OO$ contains a pure imaginary element $\psi$ which
when composed with an isomorphism $\alpha:E'\to E$ gives an isogeny
$\phi=\alpha\circ\psi:E_1\to E_2$, defined over~$K$.  For the case
$D=-4$, when $\sqrt{-1}\notin K$, we take an equation for $E$ of the
form $Y^2=X^3+aX$, set $E': Y^2=X^3-4aX$ and observe that $E$ and $E'$
are $2$-isogenous, not isomorphic over $K$ but isomorphic over
$K(\sqrt{-1})$ where $-4$ becomes a $4$th power.

\end{proof}

Hence for any isogeny class of curves which either do not have CM or
for which the extra endomorphisms are defined over $K$ itself, the
$j$-invariants are distinct and we can sort the curves using their
$j$-invariants as in Section~1.

It remains to consider isogeny classes of curves with CM by order in
an imaginary quadratic field $\Q(\sqrt{d})$ where $\sqrt{d}\notin K$.
Note that if any one curve has this property then all $K$-isogenous
curves also do, though they may have CM by different orders in
$\Q(\sqrt{d})$.

So let $E_1$ and $E_2$ be $\sqrt{d}$-twists of each other with CM by
the same order in $\Q(\sqrt{d})$.  As we have seen, they are isogenous
over $K$ and in particular have the same conductor~$\n$.  Let
$j=j(E_1)=j(E_2)$.  We may distinguish them using a prime ideal $\p$
of $K$ with certain properties.  Consider the set of primes such that
\begin{itemize}
\item $\p$ does not divide $6d\n$;
\item $\ord_{\p}(j-1728)=0$ if $j\not=1728$, otherwise
  $\ord_{\p}(j)=0$;
\item $\p$ has degree~$1$;
\item the Legendre symbol $(d/p)=-1$ where $p=N(\p)$;
\item $p$ does not divide $\disc f$ where f is the defining polynomail
  of~$K$.
\end{itemize}
The first, second and last conditions only exclude finitely many
primes, and the third one excludes a set of density~$0$; the last
condition reduces the set of admissible primes to one of density
$1/2$.  Using our ordering of primes we can let $\p$ be the
\textit{first} prime of $K$ with all the above properties, and because
of the last condition we only need use the simple form of ordering of
primes of equal norm based on the factorization of $f(X)$ modulo~$p$.
Note that the conditions we put on $\p$ are independent of the models
for $E_1$ and~$E_2$.

In the case $j\not=1728$ we can assume that $E_1$ has equation
$Y^2=X^3+aX+b$ with $b\not=0$, and $E_2: Y^2=X^3+d^2aX+d^3b$.  The
conditions on $\p$ imply that $a,b$ may be chosen so that $\p\nmid b$.
Hence exactly one of $b$, $bd^3$ is a quadratic residue modulo~$\p$,
and we can use this distinction to order $E_1$ and~$E_2$.  In order to
do this simply without changing the models, take any integral model
for $E_1$ with invariants $c_4$, $c_6$ and $\Delta$, where
$j=c_6^2/\Delta$; if $\p\nmid c_6$, which will be true unless the
model is non-minimal at~$\p$, we compute the Legendre symbol
$(c_6\pmod{\p}/p)=\pm1$ and put $E_1$ first if and only if the value
is $+1$.  In case our model is nonminimal at~$\p$ we would have
$\ord_{\p}(c_6)=6k$ and $\ord_{\p}(\Delta)=12k$ for some $k\ge1$.  Now
take $\pi$ a uniformiser at~$\p$ which has non-positive valuations at
all other primes and compute the symbol $(c_6/\pi^{6k}\pmod{\p}/p)$.
Clearly the resulting value $\pm1$ is independent of the choice of
model.

A similar analysis and formula works when $j=1728$ using $c_4$ in
place of $c_6$, dividing by $\pi^{4k}$ where $\ord_{\p}(c_4)=4k\ge0$
and $c_4$ is the invariant of any integral model for $E_1$.

This completes the task of sorting isogenous curves with the same
$j$-invariant in a model-independent way, and hence of sorting each
isogeny class of curves.

\end{document}
