\documentclass{article}
\usepackage{amsmath, amsfonts, amsthm, amssymb,fullpage}
%\usepackage{sagetex}
\usepackage{comment}
\usepackage{hyperref}
\usepackage{color}
\definecolor{mylinkcolor}{rgb}{0.5,0.0,0.0}
\definecolor{myurlcolor}{rgb}{0.0,0.0,0.75}
\hypersetup{colorlinks=true,urlcolor=myurlcolor,citecolor=myurlcolor,linkcolor=mylinkcolor,linktoc=page,breaklinks=true}

\newtheorem{thm}{Theorem}[section]
\newtheorem{cor}[thm]{Corollary}
\newtheorem{lem}[thm]{Lemma}
\newtheorem{prop}[thm]{Proposition}
\newtheorem{defn}{Definition}%[section]
\theoremstyle{remark}
\newtheorem{rem}{Remark}
\newtheorem*{example}{Example}

\def\Sage{{\tt Sage}}
\def\Pari{{\tt Pari}}
\def\Z{{\mathbb Z}}
\def\Q{{\mathbb Q}}
\def\R{{\mathbb R}}
\def\C{{\mathbb C}}
\def\T{{\mathbb T}}
\def\H{{\mathbb H}}
%\def\RC{{K_\infty}}
\def\RC{{\tilde{K}}}
\def\P{{\mathbb P}}
\def\F{{\mathbb F}}
\def\Fp{{\mathbb F}_p}
\def\Fq{{\mathbb F}_q}
\def\M{{\mathcal M}}
\def\CC{{\mathcal C}}
\def\HH{{\mathcal H}}
\def\SS{{\mathcal S}}
\def\G{{\mathcal G}}
\def\K{{\mathcal K}}
\def\I{{\mathcal I}}
\def\J{{\mathcal J}}
\def\PP{{\mathcal P}}
\def\OO{{\mathcal O}}
\def\ZG{{\mathcal Z}}
\def\a{{\mathfrak a}}
\def\b{{\mathbf b}}

\def\v{{\mathbf v}}
%\def\c{{\mathfrak c}}
\def\d{{\mathfrak d}}
\def\D{{\mathfrak D}}
\def\m{{\mathfrak m}}
\def\n{{\mathfrak n}}
\def\N{{\mathfrak N}}
\def\p{{\mathfrak p}}
\def\q{{\mathfrak q}}
\def\r{{\mathfrak r}}
\def\u{{\mathfrak u}}
\def\DF{{\nabla\underline{F}}}
\def\Qbar{\overline{\Q}}
\def\Qpbar{\overline{\Q_p}}
\def\Abar{\overline{A}}
\def\RR{{R\oplus R}}
\def\KK{{K\oplus K}}
\def\uF{{\underline{F}}}
\def\oC{\overline{\mathcal C}}
\def\oP{\overline{P}}
\def\oM{\overline{M}}
\DeclareMathOperator{\lcm}{lcm}
\DeclareMathOperator{\disc}{disc}
\DeclareMathOperator{\rk}{rank}
\DeclareMathOperator{\ord}{ord}
\DeclareMathOperator{\diag}{diag}
\DeclareMathOperator{\adj}{adj}
\DeclareMathOperator{\Mat}{Mat}
\DeclareMathOperator{\Hom}{Hom}
\DeclareMathOperator{\Res}{Res}
\DeclareMathOperator{\Gal}{Gal}
\DeclareMathOperator{\End}{End}
\DeclareMathOperator{\Aut}{Aut}
\DeclareMathOperator{\Cl}{Cl}
\def\<#1>{\left<#1\right>}
\newcommand{\mat}[4]{\left(\begin{matrix} %
                   #1 & #2 \\ #3 & #4 %
                  \end{matrix}\right)}
\newcommand{\smat}[4]{\left(\begin{smallmatrix} %
                   #1 & #2 \\ #3 & #4 %
                  \end{smallmatrix}\right)}
\DeclareMathOperator{\im}{im}

% label and link to an elliptic curve /Q (LMFDB label):
\newcommand{\lmfdbecLMFDB}[3]{\href{http://www.lmfdb.org/EllipticCurve/Q/#1.#2#3}{{\text{\rm#1.#2#3}}}}
% label and link to an elliptic curve /Q (Cremona label):
\newcommand{\lmfdbecCremona}[3]{\href{http://www.lmfdb.org/EllipticCurve/Q/#1#2#3}{{\text{\rm#1#2#3}}}}
% label and link to an elliptic curve isogeny class /Q (LMFDB label):
\newcommand{\lmfdbecisoLMFDB}[2]{\href{http://www.lmfdb.org/EllipticCurve/Q/#1.#2}{\text{\rm#1.#2}}}
% label and link to an elliptic curve isogeny class /Q (Cremona label):
\newcommand{\lmfdbecisoCremona}[2]{\href{http://www.lmfdb.org/EllipticCurve/Q/#1#2}{\text{\rm#1#2}}}

\title{Notes on sorting and labelling elliptic curves over number fields}
\author{John Cremona and Andrew V. Sutherland}

\begin{document}
\maketitle

These notes concern algorithms for sorting and labelling elliptic
curves defined over an arbitrary number field, including sorting and
labelling isogeny classes, and curves in each isogeny class.

In Section~\ref{sec:nfsort} we discuss an ordering of the elements in
a fixed number field, depending only on the polynomial used to define
the field.  In Section

\section{Sorting number field elements}
\label{sec:nfsort}
\subsection{Ordering the elements of a fixed number field}\label{sec:ordfix}
For ordering elements of $K=\Q$, we use the standard order on $\Q$ as
a subfield of~$\R$.

In general, let $K$ be a number field of degree~$d$ given in the form
$K=\Q(\alpha)\cong \Q[X]/(g(X))$, with $g(X)\in\Z[X]$ a monic
irreducible polynomial of degree~$d$.  Out of all such {defining
  polynomials} $g(X)$ we always use the \emph{reduced defining polynomial},
as defined in~\cite{CremonaPageSutherland}.

Every element $\beta\in K$ can be uniquely expressed as a polynomial
in $\alpha$ of degree at most $d-1$, say
$\beta=\sum_{i=0}^{d-1}b_i\alpha^i$, and we use the coefficient vector
$(b_0,b_1,\dots,b_{d-1})\in\Q^d$ as a sorting key, where $\Q^d$ is
sorted lexicographically.  To remove any ambiguity, this means that if
also $\gamma=\sum_{i=0}^{d-1}c_i\alpha^i$, then $\beta<\gamma$ if and
only if we have $b_i<c_i$ for the \textit{least} index~$i$ such that
$b_i\not=c_i$.

In particular, $\alpha^{d-1}<\dots<\alpha^2<\alpha<1$, and the induced
ordering on $\Q$ viewed as a subset of $K$ is the same as the standard
ordering of $\Q$.  However in general if $K$ and $L$ are number fields
with $K\subset L$ (more precisely, with an embedding $K\hookrightarrow
L$), then the embedding is not necessarily order-preserving, as we
illustrate in the next section.

\subsection{An incomplete alternative sorting method}

If $\beta$ and $\gamma$ are algebraic numbers contained in a number
field $K$ then we may order them as above, but this ordering depends
both on $K$ and on our fixed representation $K=\Q(\alpha)$.  In
particular, the order may change if we enlarge $K$; for example, in
the field $K=\Q(\alpha):=\Q[X]/(X^3-X^2+1)$ we have $\alpha-\alpha^2 <
\alpha$ (because $(0,1,-1) < (0,1,0)$), but if we embed $K$ in its
Galois closure $$L=\Q(\beta):=\Q[X]/(X^6 - 3X^5 + 5X^4 - 5X^3 + 5X^2 -
3X + 1)$$ via the map $\pi:K\to L$ defined by $\pi(\alpha) = \beta^5 -
2\beta^4 + 2\beta^3 - \beta^2 + 2\beta$, in the field $L$ we have
$\pi(\alpha-\alpha^2) > \pi(\alpha)$ (because
$(2,-4,5,-5,3,-1)>(0,2,-1,2,-2,1)$).

It would be convenient to define an ordering on $\Qbar$ that would
allow us to consistently order all algebraic numbers, independent of a
choice an ambient number field.  Every $\beta\in\Qbar$ is a root of a
unique irreducible primitive polynomial $g(X)\in\Z[X]$ with positive
leading coefficient.  Using the coefficient sequence of $g$ as a key,
we can unambiguously compare any two non-conjugate algebraic
numbers. By first sorting on $\deg(g)$ we can ensure that algebraic
numbers of lower degree come before those of higher degree.

However, to obtain a total order on $\Qbar$ we must also specify an
ordering of each set of Galois conjugates, algebraic numbers that are
roots of the same irreducible polynomial $g(X)$.  This could be done,
for example, by fixing an embedding of $\Qbar$ into $\C$ and
representing elements of $\C$ as vectors in $\R^2$ ordered
lexicographically. But to apply this ordering to elements of a number
field $K=\Q(\alpha)$ would require fixing an embedding of every such
$K$ into $\Qbar$ in a way that is compatible with respect to
inclusion.  This may be possible but it is certainly not easy to
define and implement.

On balance we decided to use the ordering defined in
\S\ref{sec:ordfix}, which allows us to compare algebraic numbers only
as elements of a chosen ambient field with a fixed defining polynomial
and generator.


\section{Sorting elliptic curves}
\label{sec:isog-curves}
\subsection{Elliptic curve labels}

Labels for elliptic curves over a fixed number field have three
components: a label for the conductor, a class label for each isogeny
class with that conductor, and a numbering of the curves in each
isogeny class: {\tt<conductor\_label>-<class\_label><number>}.  We may
also include a label for the base field to obtain a label with four
components:
{\tt<field\_label>-<conductor\_label>-<class\_label><number>}.

Over~$\Q$ itself, there have been different labelling scemes
historically, which we use instead: see the next section.

\subsubsection{LMFDB number field labels}  (See
\url{https://www.lmfdb.org/knowledge/show/nf.label}) A number field of
degree~$d$, signature $(r,s)$ and discriminant absolute value~$D$ has
label {\tt d.r.D.i} where $i$ is the index of the field in a list of
all those with the same degree, signature and discriminant in case
there are more than one.  For example, there are two totally real
quartic fields, {\tt 4.4.16448.1} and {\tt 4.4.16448.2}, with reduced
defining polynomials $x^4 - 2x^3 - 6x^2 + 2$ and $x^4 - 2x^3 - 7x^2 +
8x + 14$ respectively.

\subsubsection{Conductor labels}
Since conductors are integral ideals we can label them according to
the scheme defined in~\cite{CremonaPageSutherland}.  Each non-zero
integral ideal has a unique label of the form {\tt N.i} where $N$ is
the norm and $i$ is the index of the ideal in the sorted list of all
integral ideals of norm~$N$ (starting at~$1$).

\subsubsection{Isogeny class labels}
For a fixed conductor, each isogeny class is uniquely determined by
the traces of Frobenius $a_{\p}(E)$, since by the isogeny theorem, two
elliptic curves~$E_1$, $E_2$ defined over the same field~$K$ are
isogenous over~$K$ if and only if $a_{\p}(E_1)=a_{\p}(E_1)$ for all
but finitely many primes~$\p$.  To sort the isogeny classes for fixed
conductor $\n$, we include primes of bad reduction (dividing~$\n$),
defining $a_{\p}(E)$ in the usual way: $a_{\p}(E)=+1$ for split
multiplicative reduction, $-1$ for non-split and $0$ for additive
reduction.  Now we can sort the isogeny classes of any fixed conductor
as follows: take the sequence of integers $(a_{\p}(E))_{\p}$, with the
primes~$\p$ ordered as in~\cite{CremonaPageSutherland}, and order
these lexicographically.  Now each isogeny class of conductor~$\n$ has
an index in this ordering; we base the indices at~$0$ and convert to
base 26 using ``digits'' $a=0$, $b=1$, $\dots$, $z=25$, obtaining a
string from the list $a, b, \dots, z, ba, bb, \dots$.

\subsubsection{Sorting isogenous curves} Finally, the curves in each
isogeny class are sorted in some way and hence given a numerical
index, starting at 1, which forms the third part of the label.  In the
next section we describe methods used to sort the curves in each
isogeny class.

\begin{rem} In the LMFDB at present (May 2020) the isogeny class
labels follow this scheme for imaginary quadratic fields since there
the primes are sorted as in the previous section.  However the class
labels for elliptic curves over totally real fields are taken from the
labels of the associated Hilbert modular forms, whose ordering is not
necessary well-defined as described above.  This is a historical
accident, and may change in future.
\end{rem}
\begin{rem} Over $\Q$, the LMFDB uses two different labelling chemes
  for elliptic curves, which are referrred to as ``LMFDB labels'' and
  ``Cremona labels''.  The LMFDB labels for elliptic curves use the
  sorting described here for isogeny classes and curves i each class,
  and have the format~{\tt <conductor>.<class\_label><number>}, for
  example {\tt 11.a1}.  By contrast, Cremona labels have the
  format~{\tt <conductor><class\_label><number>}, for example {\tt
    11a1}.  Note that \lmfdbecLMFDB{11}{a}{1}
  and~\lmfdbecCremona{11}{a}{1} are different elliptic curves: there
  is only one isogeny class of elliptic curves of conductor~$11$,
  labelled \lmfdbecisoLMFDB{11}{a} or~\lmfdbecisoCremona{11}{a}
  respectively, and the three curves in this class are sorted
  differently in the two schemes.  Sinmilarly, for many conductors up
  to $230,000$, the sorting of both the isogeny classes and the curves
  in each class which led to the Cremona labels was done differently
  (and for some ranges of conductors did not keep to any logical
  schema).
\end{rem}
\begin{rem}
In the Antwerp tables of elliptic curves over~$\Q$ of conductor up
to~$200$ in \cite{AntwerpIV}, curves are labelled differently, with a
single upper case letter for each curve of fixed conductor, not
divided into isogeny classes except that isogenous curves had
consecutive letters.  See the table below for an example.  In Table~1
of the author's book \cite{JCbook2}, one can read off the Antwerp
labels corresponding to each Cremona label for conductors up to~$200$.
\end{rem}

\subsection{Sorting isogenous curves}
\subsubsection{Over \texorpdfstring{$\Q$}{\bf Q}}

Every elliptic curve over $\Q$ has a unique Weierstrass model which is
integral, globally minimal and with coefficients $a_1,a_2\in\{0,1\}$
and $a_3\in\{-1,0,1\}$.  We sort the curves using the
vector~$(a_1,a_2,a_3,a_4,a_6)$ of coefficients of such a model.
Moreover, within an isogeny class the triple $(a_1,a_2,a_3)$ is
constant (see \cite{Siksek}), so it suffices to sort on $(a_4,a_6)$.

This is the sorting used for the LMFDB labels for elliptic curves over
$\Q$.  It differs from the Cremona labelling, in which the ``strong
Weil'' or $\Gamma_0(N)$-optimal curve is number~$1$ and the others are
labelled in the order of being computed from the first.  At some point
the latter became completely deterministic and has been documented
elsewhere, but the details are not relevant here.

\begin{example}
The three elliptic curves of conductor~$11$ have the following labels
in the two schemas:
\[
\begin{array}{c|c|c|c}
  \text{LMFDB label} & \text{Cremona label} & \text{Antwerp IV label} & \text{equation} \\
  \hline\rule[-.3\baselineskip]{0pt}{10pt}
  \lmfdbecLMFDB{11}{a}{1} & \lmfdbecCremona{11}{a}{2} &\text{11C}& y^2 + y = x^{3} -  x^{2} - 7820 x - 263580 \\
  \lmfdbecLMFDB{11}{a}{2} & \lmfdbecCremona{11}{a}{1} &\text{11B}& y^2 + y = x^{3} -  x^{2} - 10 x - 20 \\
  \lmfdbecLMFDB{11}{a}{3} & \lmfdbecCremona{11}{a}{3} &\text{11A}& y^2 + y = x^{3} -  x^{2}
\end{array}
\]
\end{example}

\subsubsection{Over arbitrary number fields}

There are many reasons why a system of ordering isogenous curves using
Weierstrass models is hopeless over almost all number fields (the only
exceptions, perhaps, being the $7$ imaginary quadratic fields having
only $\pm1$ as units, and class number~$1$).  There may be no global
minimal model (in such cases, the LMFDB uses a ``semi-global minimal''
model which is minimal at all but one prime); even when there is a
global minimal model, scaling by units results in infinitely many such
models, even up to translation (except for imaginary quadratic fields
where the number is finite).

Instead we have devised a system of sorting isogenous curves which
does not depend on the model, based on the simple observation that in
almost all cases isogenous curves have distinct $j$-invariants, so
that we may sort the curves in an isogeny class by sorting their
$j$-invariants, as number field elements, following the schema of
Section~\ref{sec:nfsort}.  While it would be possible to use this new
system over $\Q$ also, we do not propose this as there are already two
labelling systems in use.

The basic fact we need is that in almost all cases, the elliptic
curves in an isogeny class have distinct $j$-invariants.

\begin{thm}
Let $K$ be a number field and let $E_1$, $E_2$ be elliptic curves
defined over~$K$ that are isogenous over $K$ but not isomorphic over
$K$.  Then either
\begin{enumerate}
\item $j(E_1)\not=j(E_2)$; or
\item $j(E_1)=j(E_2)$, and $E_1$ and $E_2$ both have CM by an
  imaginary quadratic order $\OO$ of discriminant~$D<0$ for which
  $\sqrt{D}\notin K$ (so that the extra endomorphisms are not defined
  over~$K$).

  In this case, $E_1$ and $E_2$ become isomorphic over
  $L=K(\sqrt{D})$, and there is a rational cyclic isogeny $\phi\colon
  E_1\to E_2$ of degree $|D|$ (if $D$ is odd), $|D|/4$ (if $D$ is even
  and $D\not=-4$) or~$2$ (if $D=-4$).
\end{enumerate}
Conversely, if $E$ is an elliptic curve defined over $K$ with CM by an
order of discriminant~$D$ not contained in~$K$, then the quadratic
twist of $E$ by $\sqrt{D}$ is isogenous but not isomorphic to~$E$
over~$K$; thus the isogeny class of $E$ consists of pairs of curves,
the curves in each pair being quadratic twists that are isomorphic
over $K(\sqrt{D})$ with CM by a quadratic order $\OO$
in~$\Q(\sqrt{D})$. The order $\OO$ is the same for each pair of
quadratic twists, but may vary (among orders in the same imaginary
quadratic field) between pairs.
\end{thm}

\begin{proof}
Let $\phi:E_1\to E_2$ be an isogeny defined over~$K$; without loss of
generality we may take $\phi$ to be cyclic, since otherwise it factors
through the multiplication-by-$m$ map $[m]$ on $E_1$ for some $m>1$.

Assume that $j(E_1)=j(E_2)$.  Then there is an isomorphism
$\alpha:E_2\to E_1$ defined over an extension $L/K$ of degree at
most~$6$, and no such isomorphism defined over $K$ itself, by
assumption.

The composite $\psi=\alpha\circ\phi$ is a cyclic endomorphism of $E_1$
of degree~${}>1$, which implies that $\End(E_1)\ne \Z$, so $E_1$ (and
therefore also $E_2$) has CM by some imaginary quadratic order~$\OO$, of
discriminant $D$, say.  Now $\psi$ is defined over $L=K(\sqrt{D})$,
and $L$ contains the field of definition of~$\alpha$, showing that
$[L:K]=2$ and that $\alpha$ is defined over~$L$.  Hence $E_1$ and
$E_2$ become isomorphic over the quadratic extension~$L$ of $K$.

Let $\sigma$ denote the nontrivial element of $\Gal(L/K)$.

Assume first that $j\not=0, 1728$, or equivalently that $D\not=-3,-4$.
Since the only nontrivial automorphism of $E_1$ is $[-1]$ it follows
that $\alpha^{\sigma}=-\alpha$, where $-\alpha=[-1]\circ\alpha$. Hence
$\psi^{\sigma}=-\psi$ (where again $-\psi=[-1]\circ\psi$).  Viewing
$\psi$ as an element of the abstract order~$\OO$, this means that
$\psi$ is ``pure imaginary''.  Moreover, the fact that $\psi$ has
cyclic kernel implies that $\psi$ is not divisible in~$\OO$ by any
integer $n>1$.  If $D$ is odd, so that $\OO=\Z[(1+\sqrt{D})/2]$, this
implies that $\psi=\pm\sqrt{D}$, so $\deg\psi=|D|$.  If $D$ is even
then $\OO=\Z[(1+\sqrt{D})/2]$ and $\psi=\pm\sqrt{D}/2$ with degree
$|D|/4$.

Suppose that $j=0$ and $D=-3$.  Without loss we may assume that $E_1$
has equation $Y^2=X^3+a$ with $a\in K^*$; since $E_2$ becomes
isomorphic to $E_1$ over the quadratic extension $K(\sqrt{-3})$, it is
isomorphic to $Y^2=X^3+a'$ where $a'/a$ represents a nontrivial
element of the kernel of the map $K^*/(K^*)^6 \to L^*/(L^*)^6$, but
the only such element (up to 6th powers) is $-27$ (easy exercise).
Hence without loss of generality $E_2$ has equation $Y^2=X^3-27a$, and
standard formulas show that $E_1$ and $E_2$ are $3$-isogenous.  To see
the last part directly, one isomorphism $E_1\to E_2$ is given by
$(x,y)\mapsto(-3x,-3\sqrt{-3}y)$ so that as in the generic case
$\alpha^{\sigma}=-\alpha$ and $\psi$ is pure imaginary, hence
$\psi=\pm\sqrt{-3}$ with degree~$3$.

Finally suppose that $j=1728$ and $D=-4$, so $L=K(\sqrt{-1})$.  The
kernel of $K^*/(K^*)^4 \to L^*/(L^*)^4$ has order~$2$ with non-trivial
element represented by~$-4$, and hence without loss of generality we
may suppose that $E_1$ has equation $Y^2=X^3+aX$ and $E_2:
Y^2=X^3-4aX$ for some $a\in K^*$.  Again it is well-known that these
are $2$-isogenous.  Alternatively, an explicit computation (similar to
that for $j=0$) shows that $\psi=\pm1\pm i$ with degree~$2$; note that
in this case $\alpha^{\sigma}\not=\pm\alpha$; instead,
$\alpha^{\sigma}=\epsilon\circ\alpha$ where $\epsilon\in\Aut(E_1)$ has
order~$4$.

For the last part, let $E$ have CM by the order $\OO$ of
discriminant~$D<0$ with $\sqrt{D}\notin K$.  Set $L=K(\sqrt{D})$ and
let $E'$ be the $\sqrt{D}$-twist of $E$.  Assume first that
$D\not=-4$; then $\OO$ contains a pure imaginary element $\psi$ which
when composed with an isomorphism $\alpha:E'\to E$ gives an isogeny
$\phi=\alpha\circ\psi:E_1\to E_2$, defined over~$K$.  For the case
$D=-4$, when $\sqrt{-1}\notin K$, we take an equation for $E$ of the
form $Y^2=X^3+aX$, set $E': Y^2=X^3-4aX$ and observe that $E$ and $E'$
are $2$-isogenous, not isomorphic over $K$ but isomorphic over
$K(\sqrt{-1})$ where $-4$ becomes a $4$th power.

\end{proof}

This theorem says that, for any isogeny class of curves defined
over~$K$ which either do not have CM or which have ``rational CM''
(meaning that the extra endomorphisms are defined over $K$ itself),
the $j$-invariants of the curves in the class are distinct; we may
therefore sort the curves using their $j$-invariants as in
Section~\ref{sec:nfsort}.  We choose this option for curves without
CM.

In the case where the curves in the isogeny class do have rational CM,
we adjust this sorting scheme as follows.  Observe that when two
elliptic curves are isogenous, if one has CM then so does the other,
and their endomorphism rings are isomorphic to orders in the same
imaginary quadratic field, though in general different orders.  In the
case of rational CM, we first sort by the absolute value of the CM
discriminant; this means that curves which have the same endomorphism
ring are grouped together.  The number of isomorphism classes in each
of these ``clusters'' of curves is equal to the class number of the
associated CM discriminant.  Within each cluster, the $j$-invariants
are distinct, though Galois conjugate, and we sort them as in
Section~1.

It remains to consider isogeny classes of curves with CM by an order
in an imaginary quadratic field $\Q(\sqrt{d})$ where $\sqrt{d}\notin
K$.  As before, if any one curve in the isogeny class has this
property, then all $K$-isogenous curves also do, though they may have
CM by different orders in $\Q(\sqrt{d})$.  As in the rational CM case,
we first sort by the absolute value of the CM discriminant, and then
by the $j$-invariant, but now there will be exactly two curves for
which these are the same, which are $\sqrt{d}$-twists of each other.

Let $E_1$ and $E_2$ be $\sqrt{d}$-twists of each other with CM by the
same order in $\Q(\sqrt{d})$.  As we have seen, they are necessarily
isogenous over $K$ and in particular have the same conductor~$\n$.
Let $j=j(E_1)=j(E_2)$.  We may distinguish the curves using a prime
ideal $\p$ of $K$ with certain properties.  Consider the set of
primes~$\p$ of $K$ lying above a rational prime~$p$, such that
\begin{enumerate}
\item $p$ does not divide $6dN_{K/\Q}(\n)$;
\item $p$ does not divide $\disc g$, where $g$ is the defining polynomial
  of~$K$;
\item the Legendre symbol $(d/p)=-1$;
\item $\p$ has degree~$1$, and hence residue field~$\F_p$;
\item $\ord_{\p}(j-1728)=0$ if $j\not=1728$, otherwise $\ord_{\p}(j)=0$;
\end{enumerate}
Conditions (1), (2) and~(5) only exclude finitely many primes; (4)
excludes a set of density~$0$; and (3) still leaves a set of primes of
$K$ of density~$1/2$.  Using our ordering of primes we can let $\p$ be
the \textit{first} prime of $K$ with all the above properties, in the
sense of~\cite{CremonaPageSutherland}.  This means that
$\p=(p,\alpha+r)$ with $r\in\{0,1,\dots,p-1\}$ and $r$ minimal. Note
that the conditions we put on $\p$ are independent of the models for
$E_1$ and~$E_2$.

In the case $j\not=1728$, we can assume that $E_1$ and $E_2$ have
equations $Y^2=X^3+aX+b$ and $Y^2=X^3+d^2aX+d^3b$ respectively, with
$b\not=0$.  The conditions on $\p$ imply that the coefficients $a,b$
of the model for $E_1$ may be chosen so that $\p\nmid b$.  Hence
exactly one of $b$, $bd^3$ is a quadratic residue modulo~$\p$, and we
can use this distinction to order $E_1$ and~$E_2$.  In order to do
this without having to change models, take any integral model for
$E_1$ with invariants $c_4$, $c_6$ and $\Delta$, where
$j-1728=c_6^2/\Delta$; if $\p\nmid c_6$, which will be true unless the
model is non-minimal at~$\p$, we compute the Legendre symbol
$(c_6\pmod{\p}/p)=\pm1$ and put $E_1$ first if and only if the value
is $+1$.  In case our model is non-minimal at~$\p$, we will have
$\ord_{\p}(c_6)=6k$ and $\ord_{\p}(\Delta)=12k$ for some $k\ge1$.  Now
take $\pi$ a uniformiser at~$\p$ which has non-positive valuations at
all other primes and compute the symbol $(c_6/\pi^{6k}\pmod{\p}/p)$.
Clearly the resulting value $\pm1$ is independent of the choice of
model.

A similar analysis works when $j=1728$ using $c_4$ in place of $c_6$,
using the value of the Legendre symbol $(c_4/\pi^{4k}\pmod{\p}/p)$
where $\ord_{\p}(c_4)=4k\ge0$ and $c_4$ is the invariant of any
integral model for $E_1$.

This completes the task of sorting isogenous curves with the same
$j$-invariant in a model-independent way, and hence of sorting each
isogeny class of curves.  In summary, we sort the curves in each
isogeny class using as key the triple $(|D|,j,\epsilon)$ attached to
each elliptic curve $E$ in the isogeny class, where
\begin{itemize}
\item $D=\disc\End(E)$ (so $D<0$ if $E$ has CM, otherwise $D=1$);
\item $j=j(E)$, ordered as in Section~1;
\item $\epsilon\in\{-1,1\}$ is minus the value of the Legendre symbol
  defined above when $E$ has CM but not rational CM, and $\epsilon=0$
  otherwise.
\end{itemize}

\section{Minimal twists of elliptic curves}

\subsection{Basic notions}
Let $E$ be an elliptic curve defined over the number field~$K$.  The
twists of $E$ are the other elliptic curves $E'$ defined over~$K$ with
$j(E)=j(E')$, or equivalently such that $E$ and~$E'$ become isomorphic
over an extension of~$K$.

When $j(E)\not=0,1728$ all twists of $E$ are \textit{quadratic twists}
$E'=E^{(d)}$ which are isomorphic to~$E$ over quadratic
extensions~$K(\sqrt{d})/K$; these are parametrized by $K^*/(K^*)^2$.
For $j=1728$ and $j=0$, respectively, we must consider
\textit{quartic} and \textit{sextic} twists, parametrized
by~$K^*/(K^*)^4$ and $K^*/(K^*)^6$.  One can handle all these cases in
one by saying that the twists of~$E$ are parametrized by
$H^1(G_K,\Aut(E))$ where $G_K=\Gal(\Qbar/K)$, but here we will use the
more elementary language since we want to be as explicit as possible.

Our aim in this section is to define a \textit{minimal quadratic
  twist} for every elliptic curve, and also a \textit{minimal twist}
to cover the two exceptional $j$-invariants.  For $j(E)\not=0,1728$
the minimal quadratic twist of $E$ depends only on $j(E)$, so our
definition involves picking out a unique well-defined representative
curve with each $j$-invariant; in other words, a representative curve
defined over~$K$ for each $\Qbar$-isomorphism class.

For minimal representatives with respect to the higher order twists,
we can simply pick out one curve with $j=0$ and one with $j=1728$. For
example, the curve with LMFDB label \lmfdbecLMFDB{27}{a}{4} (Cremona
label~\lmfdbecCremona{27}{a}{3}) has $j=0$ and minimal conductor~$27$
among all elliptic curves defined over~$\Q$ with $j(E)=0$, and minimal
discriminant $-27$ (among all curves with $j(E)=0$ and
conductor~$27$).  Similarly, \lmfdbecLMFDB{32}{a}{3} (Cremona
label~\lmfdbecCremona{32}{a}{2}) has minimal conductor $32$ and
minimal discriminant~$64$ among those with $j(E)=1728$.  Nevertheless,
we will also define minimal quadratic twists for these $j$-invariants.

In general we require that the definition of which curve is the
minimal quadratic twist should depend only on the isomorphism class of
the curves and not on properties of the specific models used.  In many
cases it will suffice to consider the norms of the conductor~$\n$ and
of the \textit{minimal discriminant ideal}~$\d$ (defined below), but
we will need to examine further the situation where more than one
quadratic twist has the same values for both of these.

Recall that the minimal discriminant ideal is the integral ideal
$\prod\p^{d_{\p}}$ whose valuation at each prime~$\p$ is that of the
discriminant of the local minimal model at~$\p$.  For every
Weierstrass model of $E$ with discriminant $\Delta\in K^*$ we have
$(\Delta)=\d\u^{12}$ for some fractional ideal~$\u$, and for every
other Weierstrass model with discriminant~$\Delta'$ we have
$\Delta'/\Delta\in (K^*)^{12}$.  Hence the class of $\Delta$ modulo
$12$th powers is a well-defined invariant of~$E$, as is the ideal
class $[\u]$ of~$\u$.  This class is trivial if and only if $E$
possesses a global minimal model.

We first consider the case~$K=\Q$.

\subsection{Minimal quadratic twists over \texorpdfstring{$\Q$}{\textbf Q}}

We start by studying the situation where two elliptic curves $E_1$
and~$E_2$ defined over~$\Q$, which are quadratic twists by $d$, have
the same \textit{minimal discriminant}~$\Delta_{\text{min}}$.

\begin{prop} Let $E_1$ and $E_2$ be elliptic curves defined over~$\Q$
  which are non-isomorphic quadratic twists with the same minimal
  discriminant.  Then $E_2$ is the $-1$-twist of~$E_1$, and
  $j(E_1)=j(E_2)\not=1728$.
\end{prop}
\begin{proof}
Suppose that $E_2$ is the $d$-twist of $E_1$, where without loss of
generality $d$ is square-free, and $d\not=1$.  For a suitable choice
of Weierstrass models, the discriminants of~$E_1$ and~$E_2$ are
$\Delta_1$ and $\Delta_2=d^6\Delta_1$.  Since both sides are
well-defined modulo $12$th powers and
$\Delta_1\equiv\Delta_2\pmod{(\Q^*){}^{12}}$, it follows that $d^6$ is a
$12$th power, which implies that $d=\pm1$ (since $d$ is square-free)
and so $d=-1$.  For the last part, if $j(E_1)=j(E_2)=1728$, then with
the same models the ratio of the $c_4$-invariants is $d^2=1$, giving
$E_1\cong E_2$.  (In this case, $E_1$ is its own $-1$-twist.)
\end{proof}

Now fix a rational $j$-invariant $j$.  Among all elliptic curves $E$
defined over~$\Q$ with $j(E)=j$, there is a finite subset with minimal
conductor~$N$, and a subset of those for which
$\left|\Delta_{\text{min}}\right|$ is minimal.  If this subset
contains precisely one isomorphism class of curves, we define that to
be the minimal quadratic twist.  Otherwise, the proposition shows that
$j\not=1728$ and the subset contains precisely two curves which are
$-1$-twists of each other.  Now the ratio of their $c_6$-invariants,
which is well-defined modulo $6$th powers, is $(-1)^3=-1$, so exactly
one of the two curves has positive $c_6$, and we choose this one as
the minimal quadratic twist.

\begin{rem}
An alternative way to break the tie in the last situation, which is
similar to the system proposed above for sorting isogenous quadratic
twists, would be to take the first prime $p>3$ of good reduction and
congruent to~$3\pmod4$ and pick the curve for which $(c_6/p)=+1$.  The
method proposed here is simpler.
\end{rem}

In practice we can find the minimal quadratic twist of a curve $E$ as
follows.  Start with an integral short Weierstrass model
$Y^2=X^3+aX+b$ for $E$.  For each prime~$p$ dividing $\gcd(a,b)$, let
$k$ be maximal such that $p^{2k}\mid a$ and $p^{3k}\mid b$, and
replace $(a,b)$ by~$(a/p^{2k},b/p^{3k})$.  This gives a curve~$E_1$
which it is easy to see is the minimal twist of~$E$ at all
primes~$p>3$.  Now compute the $d$-twists of~$E_1$ for
$d\in\{\pm1,\pm2,\pm3,\pm6\}$, or just $d\in\{1,2,3,6\}$ if
$j(E)=1728$, and discard first those whose conductor is not minimal,
and then those whose minimal discriminant is non-minimal among the
curves still remaining. This will leave either one or two curves
remaining; in the latter case exactly one will have positive $c_6$, we
discard the other, and are left with the minimal quadratic twist.

\bibliographystyle{abbrv}
\begin{thebibliography}{10}
\providecommand{\MR}{\relax\ifhmode\unskip\space\fi MR }
\bibitem{AntwerpIV}
B.~J. Birch and W.~Kuyk (eds.), \emph{Modular functions of one variable. {IV}},
  Springer-Verlag, Berlin, 1975, Lecture Notes in Mathematics, Vol. 476. \MR{51
  \#12708}
\bibitem{JCbook2}
J.~E. Cremona.
\newblock {\em Algorithms for Modular Elliptic Curves}.
\newblock Cambridge University Press, second edition, 1997.
\newblock available from
  \url{http://www.warwick.ac.uk/staff/J.E.Cremona/book/fulltext/index.html}.
\bibitem{Siksek} Samir Siksek, {\em On standardized models of
  isogenous elliptic curves}. Math. Comp. 74 (2005), no. 250, 949--951.
\bibitem{CremonaPageSutherland} John Cremona, Aurel Page, and Andrew
  V. Sutherland, {\em Sorting and labelling integral ideals in a
    number field}, preprint.
\end{thebibliography}
\end{document}
